\documentclass{article}
\usepackage{amsmath,amsfonts,amsthm,amssymb}
\usepackage{setspace}
\usepackage{fancyhdr}
\usepackage{lastpage}
\usepackage{extramarks}
\usepackage{chngpage}
\usepackage{soul,color}
\usepackage{graphicx,float,wrapfig}
\newcommand{\Class}{Algorithm for NLP}
\newcommand{\ClassInstructor}{Dr. Alon Lavie, Dr. Noah Smith and Dr. Bob Frederking}

% Homework Specific Information. Change it to your own
\newcommand{\Title}{Homework 4b}
\newcommand{\DueDate}{November 24, 2013}
\newcommand{\StudentName}{Bowen Low}
\newcommand{\AndrewID}{blow}

% In case you need to adjust margins:
\topmargin=-0.45in      %
\evensidemargin=0in     %
\oddsidemargin=0in      %
\textwidth=6.5in        %
\textheight=9.0in       %
\headsep=0.25in         %

% Setup the header and footer
\pagestyle{fancy}                                                       %
\lhead{\StudentName}                                                 %
\chead{\Title}  %
\rhead{\firstxmark}                                                     %
\lfoot{\lastxmark}                                                      %
\cfoot{}                                                                %
\rfoot{Page\ \thepage\ of\ \protect\pageref{LastPage}}                          %
\renewcommand\headrulewidth{0.4pt}                                      %
\renewcommand\footrulewidth{0.4pt}                                      %

%%%%%%%%%%%%%%%%%%%%%%%%%%%%%%%%%%%%%%%%%%%%%%%%%%%%%%%%%%%%%
% Some tools
\newcommand{\enterProblemHeader}[1]{\nobreak\extramarks{#1}{#1 continued on next page\ldots}\nobreak%
                                    \nobreak\extramarks{#1 (continued)}{#1 continued on next page\ldots}\nobreak}%
\newcommand{\exitProblemHeader}[1]{\nobreak\extramarks{#1 (continued)}{#1 continued on next page\ldots}\nobreak%
                                   \nobreak\extramarks{#1}{}\nobreak}%

\newcommand{\homeworkProblemName}{}%
\newcounter{homeworkProblemCounter}%
\newenvironment{homeworkProblem}[1][Problem \arabic{homeworkProblemCounter}]%
  {\stepcounter{homeworkProblemCounter}%
   \renewcommand{\homeworkProblemName}{#1}%
   \section*{\homeworkProblemName}%
   \enterProblemHeader{\homeworkProblemName}}%
  {\exitProblemHeader{\homeworkProblemName}}%

\newcommand{\homeworkSectionName}{}%
\newlength{\homeworkSectionLabelLength}{}%
\newenvironment{homeworkSection}[1]%
  {% We put this space here to make sure we're not connected to the above.

   \renewcommand{\homeworkSectionName}{#1}%
   \settowidth{\homeworkSectionLabelLength}{\homeworkSectionName}%
   \addtolength{\homeworkSectionLabelLength}{0.25in}%
   \changetext{}{-\homeworkSectionLabelLength}{}{}{}%
   \subsection*{\homeworkSectionName}%
   \enterProblemHeader{\homeworkProblemName\ [\homeworkSectionName]}}%
  {\enterProblemHeader{\homeworkProblemName}%

   % We put the blank space above in order to make sure this margin
   % change doesn't happen too soon.
   \changetext{}{+\homeworkSectionLabelLength}{}{}{}}%

\newcommand{\Answer}{\ \\\textbf{Answer:} }
\newcommand{\Acknowledgement}[1]{\ \\{\bf Acknowledgement:} #1}

%%%%%%%%%%%%%%%%%%%%%%%%%%%%%%%%%%%%%%%%%%%%%%%%%%%%%%%%%%%%%


%%%%%%%%%%%%%%%%%%%%%%%%%%%%%%%%%%%%%%%%%%%%%%%%%%%%%%%%%%%%%
% Make title
\title{\textmd{\bf \Class: \Title}\\{\large Instructed by \textit{\ClassInstructor}}\\\normalsize\vspace{0.1in}\small{Due\ on\ \DueDate}}
\date{}
\author{\textbf{\StudentName}\ \ \AndrewID}
%%%%%%%%%%%%%%%%%%%%%%%%%%%%%%%%%%%%%%%%%%%%%%%%%%%%%%%%%%%%%

\begin{document}
\begin{spacing}{1.1}
\maketitle \thispagestyle{empty}

%%%%%%%%%%%%%%%%%%%%%%%%%%%%%%%%%%%%%%%%%%%%%%%%%%%%%%%%%%%%%
% Begin edit from here

\begin{homeworkProblem}[Derivation Semirings]
\Answer 
\begin{itemize}
\item{Semiring 4} \\
0: It is a 2-tuple with the integer 0 in the first position and an empty set in the second position. \\
1: It is a 2-tuple with the integer 1 in the first position and an empty set in the second position. \\
SemiPlus: It performs the union between 2 2-tuples.  \\
SemiTimes: It performs the union between 2 2-tuples. \\
Space: There could be $XL^2R$  \\
\item{Semiring 5} \\
0: It is a 3-tuple with the integer 0 in the first position, and a empty tuple in the second 
position and an empty string in the third position. \\
1: It is a 3-tuple with the integer 1 in the first position, and empty tuple in the second 
position and an empty string in the third position. \\
SemiPlus: Given 2 3-tuples, it checks which has the higher value in the 1st position. It will
return the 3-tuple with the higher value in the 1st position. \\
SemiTimes: Given 2 3-tuples $a$ and $b$, it will return a 3-tuple where the first value is the 
result of mutiplying the numerical value of both 3-tuples. The tuple returned is the tuple
of $a$ less the first item. The string returned is the concatenation of the string from $a$ to the
string from $b$. If the string from $b$ is from the right child, then parentheses are added around
the string. \\
Space: $O(XL^3(N+W))$ \\ 
Time: $O(X^3*L^3)$. \\
\item{Semiring 6} \\
0: It is an array, containing a single semiZero from SemiRing 5. \\
1: It is an array, containing a single semiOne from SemiRing 5. \\
SemiPlus: Given 2 arrays $a$ and $b$, it performs a concatenation of the two arrays, then it 
sorts the new array by the numerical values in each 3-tuple and retains the top 10. It then
returns the array of 10 items. \\
SemiTimes: Given 2 arrays $a$ and $b$, for each element in $a$, it performs the semiTimes of 
Semiring 5 for every element in $b$. Then given the new array containing all resulting 3-tuples,
it sorts the new array by the numerical values in each 3-tuple and retains the top 10. It then
returns the array of 10 items.  \\
Space: $O(L^2K)$. 
\item{Semiring 7} \\
0: It is a 3-tuple, with the integer 0 in the first position, a 3-tuple in the 2nd position, and 
an empty array in the third position. As for the 3-tuple in the 2nd position, it has an empty 
string in the first position, integer max in the 2nd and 0 in the third. This represents the 
word for a given span. The 2nd integer represents the start position and the 3rd integer 
represents the end position. \\
1: It is a 3-tuple, with the integer 1 in the first position, a 3-tuple in the 2nd position and 
an empty array in the third position. As for the 3-tuple in the 2nd position, it has an empty
string in the first position, integer max in the 2nd and 0 in the third position. This 
represents the word for a given span. The 2nd integer represents the start position and the 
3rd position \\
SemiPlus: Given 2 3-tuples $a$ and $b$, it returns the 3-tuple with the higher weight value 
in the first position. \\
SemiTimes: Given 2 3-tuples $a$ and $b$, it returns the 3-tuple where the the 1st position 
stores the multiplicative result of the numerical values in $a$ and $b$. The array in the 3rd
position stores the result after appending the word in $b$ to the array in $a$. The word in the 
new 3-tuple stores the word from $a$, the minimum of the two start positions and the maximum of
the two end positions.  \\
Space: $O(XL^2)$
\end{itemize}
\end{homeworkProblem}

\begin{homeworkProblem}[Reverse CNF Transformation]
\Answer 
No transforming the parse trees is sufficient. As the transformation of the Grammar to CNF is 
a deterministic process, the reverse transformation is also deterministic and therefore given 
a parse tree in CNF grammar, there can only be 1 tree after the reverse CNF process and therefore
there is no need to recalculate the derivation scores.
\end{homeworkProblem}

\begin{homeworkProblem}[Agenda Pruning]
\Answer
\begin{enumerate}
\item{}
No. Agenda pruning depends on the heuristic used, and since heuristics cannot guarantee
correctness, therefore it cannot guarantee to produce a parse if it exists. 
\item{}
Yes. Because the value of agenda items are calculated in a bottom-up manner items retain their 
values as if none of their children were pruned (for a rule to be instantiated both children 
have to be present) therefore any score calculated would be correct.
\item{}
SENT 0 AGENDA ADDS: 387 \\
SENT 0 GOAL SCORE: None \\
SENT 1 AGENDA ADDS: 31 \\
SENT 1 GOAL SCORE: 2.28006028891e-06 \\
SENT 2 AGENDA ADDS: 150 \\
SENT 2 GOAL SCORE: None \\
SENT 3 AGENDA ADDS: 63 \\
SENT 3 GOAL SCORE: None \\
SENT 4 AGENDA ADDS: 253 \\
SENT 4 GOAL SCORE: None \\
SENT 5 AGENDA ADDS: 179 \\
SENT 5 GOAL SCORE: None \\
SENT 6 AGENDA ADDS: 54 \\
SENT 6 GOAL SCORE: 5.73824670126e-09 \\
SENT 7 AGENDA ADDS: 49 \\
SENT 7 GOAL SCORE: 1.25924395682e-08 \\
SENT 8 AGENDA ADDS: 100 \\
SENT 8 GOAL SCORE: None \\
SENT 9 AGENDA ADDS: 251 \\
SENT 9 GOAL SCORE: None \\
SENT 10 AGENDA ADDS: 318 \\
SENT 10 GOAL SCORE: None \\
No I did not produce parses for all sentences. No the final scores are not identical to the true
scores. Pruning with a constant threshold is not a good idea as scores tend to decrease higher
up the parse tree as more rule weights are multiplied together and since for this case the scores
are probabilities, the score value would become very small near the root. A constant threshold 
would not take into account the changing values and remove agenda items in a long sentence.
\item{}
SENT 0 AGENDA ADDS: 456 \\
SENT 0 GOAL SCORE: 3.06412448871e-41 \\
SENT 1 AGENDA ADDS: 31 \\
SENT 1 GOAL SCORE: 2.28006028891e-06 \\
SENT 2 AGENDA ADDS: 185 \\
SENT 2 GOAL SCORE: 1.46131493773e-19 \\
SENT 3 AGENDA ADDS: 64 \\
SENT 3 GOAL SCORE: 4.52096543091e-13 \\
SENT 4 AGENDA ADDS: 345 \\
SENT 4 GOAL SCORE: 3.77079612505e-27 \\
SENT 5 AGENDA ADDS: 198 \\
SENT 5 GOAL SCORE: 2.20933417212e-15 \\
SENT 6 AGENDA ADDS: 54 \\
SENT 6 GOAL SCORE: 5.73824670126e-09 \\
SENT 7 AGENDA ADDS: 49 \\
SENT 7 GOAL SCORE: 1.25924395682e-08 \\
SENT 8 AGENDA ADDS: 105 \\
SENT 8 GOAL SCORE: 8.33570429924e-14 \\
SENT 9 AGENDA ADDS: 330 \\
SENT 9 GOAL SCORE: 2.21661096379e-22 \\
SENT 10 AGENDA ADDS: 386 \\
SENT 10 GOAL SCORE: 3.85139619145e-36 \\
Yes parses were produced for all sentences. The final scores are identical to true scores without
pruning. 
\end{enumerate}
\end{homeworkProblem}

\Acknowledgement{}

% End edit to here
%%%%%%%%%%%%%%%%%%%%%%%%%%%%%%%%%%%%%%%%%%%%%%%%%%%%%%%%%%%%%

\end{spacing}
\end{document}

%%%%%%%%%%%%%%%%%%%%%%%%%%%%%%%%%%%%%%%%%%%%%%%%%%%%%%%%%%%%%
